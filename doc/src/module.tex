\chapter{Module system}
\label{chapter:modules}

This chapter describes Scheme~48's module system.
The module system is unique in the extent to which it
supports both static linking and rapid turnaround during program
development.  The design was influenced by Standard ML
modules\cite{MacQueen:Modules} and by the module system for Scheme
Xerox\cite{Curtis-Rauen:Modules}.  It has also been shaped by the
needs of \hack{}, which is designed to run both on workstations and
on relatively small (less than 1 Mbyte) embedded controllers.

Except where noted, everything described here is implemented in
\hack{}, and exercised by the \hack{} implementation and some
application programs.

Unlike the Common Lisp package system, the module system described
here controls the mapping of names to denotations, not the
mapping of strings to symbols.

\section{Introduction}

The module system supports the structured division of a corpus of
Scheme software into a set of modules.  Each module has its own
isolated namespace, with visibility of bindings controlled by module
descriptions written in a special {\em configuration language.}

A module may be instantiated multiple times, producing several {\em
packages}, just as a lambda-expression can be instantiated multiple
times to produce several different procedures.  Since single
instantiation is the normal case, we will defer discussion of multiple
instantiation until a later section.  For now you can think of a
package as simply a module's internal environment mapping names to
denotations.

A module exports bindings by providing views onto the underlying
package.  Such a view is called a {\em structure} (terminology from
Standard ML).  One module may provide several different views.  A
structure is just a subset of the package's bindings.  The particular
set of names whose bindings are exported is the structure's {\em
interface}.

A module imports bindings from other modules by either {\em opening}
or {\em accessing} some structures that are built on other packages.
When a structure is opened, all of its exported bindings are visible
in the client package.
%On the other hand, bindings from an accessed
%structure require explicitly qualified references written with the
%{\tt structure-ref} operator (see below).

For example:
\begin{example}
(define-structure foo (export a c cons)
  (open scheme)
  (begin (define a 1)
         (define (b x) (+ a x))
         (define (c y) (* (b a) y))))

(define-structure bar (export d)
  (open scheme foo)
  (begin (define (d w) (+ a (c w)))))
\end{example}
This configuration defines two structures, {\tt foo} and {\tt bar}.
{\tt foo} is a view on a package in which the {\tt scheme} structure's
bindings (including {\tt define} and {\tt +}) are visible, together
with bindings for {\tt a}, {\tt b},
and {\tt c}.  {\tt foo}'s interface is {\tt (export a c cons)}, so of
the bindings in its underlying package, {\tt foo} only exports those
three.  Similarly, structure {\tt bar} consists of the binding of {\tt
d} from a package in which both {\tt scheme}'s and {\tt foo}'s
bindings are visible.  {\tt foo}'s binding of {\tt cons} is imported
from the Scheme structure and then re-exported.

A module's body, the part following {\tt begin} in the above example,
is evaluated in an isolated lexical scope completely specified by the
package definition's {\tt open} and {\tt access} clauses.  In
particular, the binding of the syntactic operator {\tt define-structure}
is not visible unless it comes from some opened structure.  Similarly,
bindings from the {\tt scheme} structure aren't visible unless they
become so by {\tt scheme} (or an equivalent structure) being opened.


\section{The configuration language}
%\renewcommand{\topfraction}{0.8}

The configuration language consists of top-level defining forms for
 modules and interfaces.
Its syntax is given in figure~\ref{module-language-figure}.

\texonly\setbox0\hbox{\goesto}\endtexonly
\texonly\newcommand{\altz}{\hbox to 1\wd0{\hfil\alt}}\endtexonly
\htmlonly\newcommand{\altz}{\goesto}\endhtmlonly

\begin{figure}[htb]

  \begin{tabular}{l}
   \syn{configuration} \goesto{}~\arbno{\syn{definition}} \\

   \begin{tabular}{rl}
   \syn{definition} \goesto{} &
      \tt(define-structure \syn{name} \syn{interface}
          \arbno{\syn{clause}}) \\
        \altz{} & \tt(define-structures (\arbno{(\syn{name} \syn{interface})})
          \arbno{\syn{clause}}) \\
        \altz{} & \tt(define-interface \syn{name} \syn{interface}) \\
        \altz{} & \tt(define-syntax \syn{name} \syn{transformer-spec})
   \end{tabular}
   \\
   \begin{tabular}{rl}
   \syn{clause} \goesto{} & \tt(open \arbno{\syn{structure}}) \\
        \altz{} & \tt(access \arbno{\syn{name}}) \\
        \altz{} & \tt(begin \syn{program}) \\
        \altz{} & \tt(files \arbno{\syn{filespec}}) \\
        \altz{} & \tt(optimize \arbno{\syn{optimize-spec}}) \\
        \altz{} & \tt(for-syntax \arbno{\syn{clause}})
   \end{tabular}
   \\
   \begin{tabular}{rl}
   \syn{interface} \goesto{} &  \tt(export \arbno{\syn{item}}) \\
        \altz{} & \syn{name} \\
        \altz{} & \tt(compound-interface \arbno{\syn{interface}})
   \end{tabular}
   \\
   \begin{tabular}{rl}
   \syn{item} \goesto{}~ \syn{name} \\
        \altz{} & \tt(\syn{name} \syn{type}) \\
        \altz{} & \tt((\arbno{\syn{name}}) \syn{type})
   \end{tabular}
   \\
   \begin{tabular}{rl}
   \syn{structure} \goesto{} & \syn{name} \\
        \altz{} & \tt(modify \syn{structure} \arbno{\syn{modifier}}) \\
        \altz{} & \tt(subset \syn{structure} (\arbno{\syn{name}})) \\
        \altz{} & \tt(with-prefix \syn{structure} \syn{name})
   \end{tabular}
   \\
   \begin{tabular}{rl}
   \syn{modifier} \goesto{} & \tt(expose \arbno{\syn{name}}) \\
        \altz{} & \tt(hide \arbno{\syn{name}}) \\
        \altz{} & \tt(rename \arbno{(\syn{name}$_0$ \syn{name}$_1$)}) \\
        \altz{} & \tt(alias \arbno{(\syn{name}$_0$ \syn{name}$_1$)}) \\
        \altz{} & \tt(prefix \syn{name})
   \end{tabular}
 \end{tabular}
 \caption{The configuration language.}
\label{module-language-figure}
\end{figure}

A \codemainindex{define-structure} form introduces a binding of a name to a
structure.  A structure is a view on an underlying package which is
created according to the clauses of the {\tt define-structure} form.
Each structure has an interface that specifies which bindings in the
structure's underlying package can be seen via that structure in other
packages.

An {\tt open} clause specifies which structures will be opened up for
 use inside the new package.
At least one structure must be specified or else it will be impossible to
 write any useful programs inside the package, since {\tt define},
 {\tt lambda}, {\tt cons}, etc.\ will be unavailable.
Packages typically include {\tt scheme}, which exports all bindings
 appropriate to Revised$^5$ Scheme, in an {\tt open} clause.
For building structures that export structures, there is a {\tt defpackage}
 package that exports the operators of the configuration language.
Many other structures, such as record and hash table facilities, are also
 available in the \hack{} implementation.

The \codemainindex{modify}, \codemainindex{subset}, and
 \codemainindex{prefix} forms produce new
 views on existing structures by renaming or hiding exported names.
\code{Subset} returns a new structure that exports only the listed names
 from its \syn{structure} argument.
\code{With-prefix} returns a new structure that adds \syn{prefix}
 to each of the names exported by the \syn{structure} argument.
For example, if structure \code{s} exports \code{a} and \code{b},
 then
\begin{example}
(subset s (a))
\end{example}
 exports only \code{a} and
\begin{example}
(with-prefix s p/)
\end{example}
exports \code{a} as \code{p/a} and \code{b} as \code{p/b}.

Both \code{subset} and \code{with-prefix} are simple macros that
 expand into uses of \code{modify}, a more general renaming form.
In a \code{modify} structure specification the \syn{command}s are applied to
 the names exported
 by \syn{structure} to produce a new set of names for the \syn{structure}'s
 bindings.
\code{Expose} makes only the listed names visible.
\code{Hide} makes all but the listed names visible.
\code{Rename} makes each \syn{name}$_0$ visible as \syn{name}$_1$ 
 name and not visible as \syn{name}$_0$ , while
 \code{alias} makes each \syn{name}$_0$ visible as both \syn{name}$_0$ 
 and \syn{name}$_1$.
\code{Prefix} adds \syn{name} to the beginning of each exported name.
The modifiers are applied from right to left.  Thus
\begin{example}
(modify scheme (prefix foo/) (rename (car bus))))
\end{example}
 makes \code{car} available as \code{foo/bus}.

% Use modify instead of structure-ref.
%
%An {\tt access} clause specifies which bindings of names to structures
%will be visible inside the package body for use in {\tt structure-ref}
%forms.  {\tt structure-\ok{}ref} has the following syntax:
%\begin{tabbing}
%\qquad \syn{expression} \goesto{}~
%   \tt(structure-ref \syn{struct-name} \syn{name})
%\end{tabbing}
%The \syn{struct-name} must be the name of an {\tt access}ed structure,
%and \syn{name} must be something that the structure exports.  Only
%structures listed in an {\tt access} clause are valid in a {\tt
%structure-ref}.  If a package accesses any structures, it should
%probably open the {\tt structure-refs} structure so that the {\tt
%structure-ref} operator itself will be available.

The package's body is specified by {\tt begin} and/or {\tt files}
clauses.  {\tt begin} and {\tt files} have the same semantics, except
that for {\tt begin} the text is given directly in the package
definition, while for {\tt files} the text is stored somewhere in the
file system.  The body consists of a Scheme program, that is, a
sequence of definitions and expressions to be evaluated in order.  In
practice, we always use {\tt files} in preference to {\tt begin}; {\tt
begin} exists mainly for expository purposes.

A name's imported binding may be lexically overridden or {\em shadowed}
by defining the name using a defining form such as {\tt define}
or {\tt define-\ok{}syntax}.  This will create a new binding without having
any effect on the binding in the opened package.  For example, one can
do {\tt(define car 'chevy)} without affecting the binding of the name
{\tt car} in the {\tt scheme} package.

Assignments (using {\tt set!})\ to imported and undefined variables
are not allowed.  In order to {\tt set!}\ a top-level variable, the
package body must contain a {\tt define} form defining that variable.
Applied to bindings from the {\tt scheme} structure, this restriction
is compatible with the requirements of the Revised$^5$ Scheme report.

It is an error for two of a package's opened structures to export two
different bindings for the same name.  However, the current
implementation does not check for this situation; a name's binding is
always taken from the structure that is listed first within the {\tt
open} clause.  This may be fixed in the future.

File names in a {\tt files} clause can be symbols, strings, or lists
(Maclisp-style ``namelists'').  A ``{\tt.scm}'' file type suffix is
assumed.  Symbols are converted to file names by converting to upper
or lower case as appropriate for the host operating system.  A
namelist is an operating-system-independent way to specify a file
obtained from a subdirectory.  For example, the namelist {\tt(rts
record)} specifies the file {\tt record.scm} in the {\tt rts}
subdirectory.

If the {\tt define-structure} form was itself obtained from a file,
then file names in {\tt files} clauses are interpreted relative to the
directory in which the file containing the {\tt define-structure} form
was found.  You can't at present put an absolute path name in the {\tt
files} list.


\section{Interfaces}
\codemainindex{define-interface}

An interface can be thought of as the type of a structure.  In its
basic form it is just a list of variable names, written {\tt(export
\cvar{name} \ldots)}.  However, in place of
a name one may write {\tt (\cvar{name} \cvar{type})}, indicating the type
of \cvar{name}'s binding.
The type field is optional, except
 that exported macros must be indicated with type {\tt :syntax}.

Interfaces may be either anonymous, as in the example in the
introduction, or they may be given names by a {\tt define-interface}
form, for example
\begin{example}
(define-interface foo-interface (export a c cons))
(define-structure foo foo-interface \ldots)
\end{example}
In principle, interfaces needn't ever be named.  If an interface
had to be given at the point of a structure's use as well as at the
point of its definition, it would be important to name interfaces in
order to avoid having to write them out twice, with risk of mismatch
should the interface ever change.  But they don't.

Still, there are several reasons to use {\tt define-interface}:
\begin{enumerate}
\item It is important to separate the interface definition from the
package definitions when there are multiple distinct structures that
have the same interface --- that is, multiple implementations of the
same abstraction.

\item It is conceptually cleaner, and often useful for documentation
purposes, to separate a module's specification (interface) from its
implementation (package).

\item Our experience is that configurations that are separated into
interface definitions and package definitions are easier to read; the
long lists of exported bindings just get in the way most of the time.
\end{enumerate}

The \codemainindex{compound-interface} operator forms an interface that is the
union of two or more component interfaces.  For example,
\begin{example}
(define-interface bar-interface
  (compound-interface foo-interface (export mumble)))
\end{example}
defines {\tt bar-interface} to be {\tt foo-interface} with the name
{\tt mumble} ad\-ded.


\section{Macros}

Hygienic macros, as described in
\cite{Clinger-Rees:Macros,Clinger-Rees:R4RS}, are implemented.
Structures may export macros; auxiliary names introduced into the
expansion are resolved in the environment of the macro's definition.

For example, the {\tt scheme} structure's {\tt delay} macro 
is defined by the rewrite rule
\begin{example}
(delay \cvar{exp})  \xform  (make-promise (lambda () \cvar{exp})).
\end{example}
The variable {\tt make-promise} is defined in the {\tt scheme}
structure's underlying package, but is not exported.  A use of the
{\tt delay} macro, however, always accesses the correct definition
of {\tt make-promise}.  Similarly, the {\tt case} macro expands into
uses of {\tt cond}, {\tt eqv?}, and so on.  These names are exported
by {\tt scheme}, but their correct bindings will be found even if they
are shadowed by definitions in the client package.


\section{Higher-order modules}

There are {\tt define-module} and {\tt define} forms for
defining modules that are intended to be instantiated multiple times.
But these are pretty kludgey --- for example, compiled code isn't
shared between the instantiations --- so we won't describe them yet.
If you must know, figure it out from the following grammar.
%
\begin{center}
\begin{tabular}{rl}
    \qquad \syn{definition} \goesto{} &
    \tt(define-module (\syn{name} \arbno{(\syn{name} \syn{interface})}) \\
    &\qquad \arbno{\syn{definition}} \\
    &\qquad \syn{name}\tt) \\
    \altz{} & \tt(define \syn{name} (\syn{name} \arbno{\syn{name}}))
\end{tabular}
\end{center}
% JAR says: Different instantiations of of a module may see different
% definitions of macros and inlines.

\section{Compiling and linking}

\hack{} has a static linker that produces stand-alone heap images
from module descriptions.  The programmer specifies a particular procedure in a
particular structure to be the image's startup procedure (entry
point), and the linker traces dependency links as given by {\tt open}
and {\tt access} clauses to determine the composition of the heap
image.

There is not currently any provision for separate compilation; the
only input to the static linker is source code.  However, it will not
be difficult to implement separate compilation.  The unit of
compilation is one module (not one file).  Any opened or accessed
structures from which macros are obtained must be processed to the
extent of extracting its macro definitions.  The compiler knows from
the interface of an opened or accessed structure which of its exports
are macros.  Except for macros, a module may be compiled without any
knowledge of the implementation of its opened and accessed structures.
However, inter-module optimization may be available as an option.

The main difficulty with separate compilation is resolution of
auxiliary bindings introduced into macro expansions.  The module
compiler must transmit to the loader or linker the search path by
which such bindings are to be resolved.  In the case of the {\tt delay}
macro's auxiliary {\tt make-promise} (see example above), the loader
or linker needs to know that the desired binding of {\tt make-promise}
is the one apparent in {\tt delay}'s defining package, not in the
package being loaded or linked.

%[I need to describe structure reification.]

\section{Semantics of configuration mutation}

During program development it is often desirable to make changes to
packages and interfaces.  In static languages it may be necessary to
recompile and re-link a program in order for such changes to be
reflected in a running system.  Even in interactive Common Lisp
implementations, a change to a package's exports often requires
reloading clients that have already mentioned names whose bindings
change.  Once {\tt read} resolves a use of a name to a symbol, that
resolution is fixed, so a change in the way that a name resolves to a
symbol can only be reflected by re-{\tt read}ing all such references.

The \hack{} development environment supports rapid turnaround in
modular program development by allowing mutations to a program's
configuration, and giving a clear semantics to such mutations.  The
rule is that variable bindings in a running program are always
resolved according to current structure and interface bindings, even
when these bindings change as a result of edits to the configuration.
For example, consider the following:
\begin{example}
(define-interface foo-interface (export a c))
(define-structure foo foo-interface
  (open scheme)
  (begin (define a 1)
         (define (b x) (+ a x))
         (define (c y) (* (b a) y))))
(define-structure bar (export d)
  (open scheme foo)
  (begin (define (d w) (+ (b w) a))))
\end{example}
This program has a bug.  The variable {\tt b}, which is free in the
definition of {\tt d}, has no binding in {\tt bar}'s package.  Suppose
that {\tt b} was supposed to be exported by {\tt foo}, but was omitted
from {\tt foo-interface} by mistake.  It is not necessary to
re-process {\tt bar} or any of {\tt foo}'s other clients at this point.
One need only change {\tt foo-interface} and inform the development
system of that change (using, say, an appropriate Emacs command),
and {\tt foo}'s binding of {\tt b} will be found when procedure {\tt
d} is called.

Similarly, it is also possible to replace a structure; clients of the
old structure will be modified so that they see bindings from the new
one.  Shadowing is also supported in the same way.  Suppose that a
client package $C$ opens a structure {\tt foo} that exports a name
{\tt x}, and {\tt foo}'s implementation obtains the binding of {\tt x}
as an import from some other structure {\tt bar}.  Then $C$ will see
the binding from {\tt bar}.  If one then alters {\tt foo} so that it
shadows {\tt bar}'s binding of {\tt x} with a definition of its own,
then procedures in $C$ that reference {\tt x} will automatically see
{\tt foo}'s definition instead of the one from {\tt bar} that they saw
earlier.

This semantics might appear to require a large amount of computation
on every variable reference: The specified behavior requires scanning
the package's list of opened structures, examining their interfaces,
on every variable reference, not just at compile time.  However, the
development environment uses caching with cache invalidation to make
variable references fast.


\section{Command processor support}
\label{module-commands}

While it is possible to use the \hack{} static linker for program
development, it is far more convenient to use the development
environment, which supports rapid turnaround for program changes.  The
programmer interacts with the development environment through a {\em
command processor}.  The command processor is like the usual Lisp
read-eval-print loop in that it accepts Scheme forms to evaluate.
However, all meta-level operations, such as exiting the Scheme system
or requests for trace output, are handled by {\em commands,} which are
lexically distinguished from Scheme forms.  This arrangement is
borrowed from the Symbolics Lisp Machine system, and is reminiscent of
non-Lisp debuggers.  Commands are a little easier to type than Scheme
forms (no parentheses, so you don't have to shift), but more
importantly, making them distinct from Scheme forms ensures that
programs' namespaces aren't cluttered with inappropriate bindings.
Equivalently, the command set is available for use regardless of what
bindings happen to be visible in the current program.  This is
especially important in conjunction with the module system, which puts
strict controls on visibility of bindings.

The \hack{} command processor supports the module system with a
variety of special commands.  For commands that require structure
names, these names are resolved in a designated configuration package
that is distinct from the current package for evaluating Scheme forms
given to the command processor.  The command processor interprets
Scheme forms in a particular current package, and there are commands
that move the command processor between different packages.

Commands are introduced by a comma ({\tt,}) and end at the end of
line.  The command processor's prompt consists of the name of the
current package followed by a greater-than ({\tt>}).

\begin{description}
\item \code{,open \arbno{\cvar{structure}}} \\
    The {\tt,open} command opens new structures in the current
    package, as if the package's definition's {\tt open} clause
    had listed \cvar{structure}.
    As with {\tt open} clauses the visible names can be modified,
    as in
\begin{example}
,open (subset foo (bar baz))
\end{example}
    which only makes the \code{bar} and \code{baz} bindings from
    structure \code{foo} visible.

\item \code{,config} \\
    The {\tt,config} command sets the command processor's current
    package to be the current configuration package.  Forms entered at
    this point are interpreted as being configuration language forms,
    not Scheme forms.

\item \code{,config \cvar{command}} \\
    This form of the {\tt,config} command executes another command in
    the current configuration package.  For example,
\begin{example}
,config ,load foo.scm
\end{example}
    interprets configuration language forms from the file {\tt
    foo.scm} in the current configuration package.

\item \code{,config-package-is \cvar{struct-name}} \\
    The {\tt,config-package-is} command designates a new configuration
    package for use by the {\tt,config} command and resolution of
    \cvar{struct-name}s for other commands such as {\tt,in} and
    {\tt,open}.  See
    section~\ref{config-packages}
    for information on making new configuration packages.

\item \code{,in \cvar{struct-name}} \\
    The {\tt ,in} command moves the command processor to a specified
    structure's underlying package.  For example:
\begin{example}
user> ,config
config> (define-structure foo (export a)
          (open scheme))
config> ,in foo
foo> (define a 13)
foo> a
13
\end{example}
    In this example the command processor starts in a package called
    {\tt user}, but the {\tt ,config} command moves it into the
    configuration package, which has the name {\tt config}.  The {\tt
    define-structure} form binds, in {\tt config}, the name {\tt foo} to
    a structure that exports {\tt a}.  Finally, the command {\tt ,in
    foo} moves the command processor into structure {\tt foo}'s
    underlying package.

    A package's body isn't executed (evaluated) until the package is
    {\em loaded}, which is accomplished by the {\tt ,load-package}
    command.

\item \code{,in \cvar{struct-name} \cvar{command}} \\
    This form of the {\tt,in} command executes a single command in the
    specified package without moving the command processor into that
    package.  Example:
\begin{example}
,in mumble (cons 1 2)
,in mumble ,trace foo
\end{example}

\item \code{,user [\cvar{command}]} \\
    This is similar to the {\tt ,config} and {\tt ,in} commands.  It
    moves to or executes a command in the user package (which is the
    default package when the \hack{} command processor starts).

\item \code{,user-package-is \cvar{name}} \\
    The {\tt,user-package-is} command designates a new user
    package for use by the {\tt,user} command.

\item \code{,load-package \cvar{struct-name}} \\
    The {\tt,load-package} command ensures that the specified structure's
    underlying package's program has been loaded.  This 
    consists of (1) recursively ensuring that the packages of any
    opened or accessed structures are loaded, followed by (2)
    executing the package's body as specified by its definition's {\tt
    begin} and {\tt files} forms.
%  (Note that 
%   commands related to packages, such as \code{load-package}, take structures
%   as arguments.
%  This is because packages are usually anonymous while structures are
%   usually named and each has a unique package associated with it.)

\item \code{,reload-package \cvar{struct-name}} \\
    This command re-executes the structure's package's program.  It
    is most useful if the program comes from a file or files, when
    it will update the package's bindings after mutations to its
    source file.

\item \code{,load \cvar{filespec} \ldots} \\
    The {\tt,load} command executes forms from the specified file or
    files in the current package.  {\tt,load \cvar{filespec}} is similar
    to {\tt(load "\cvar{filespec}")}
    except that the name {\tt load} needn't be bound in the current
    package to Scheme's {\tt load} procedure.

\item \code{,for-syntax [\cvar{command}]} \\
    This is similar to the {\tt ,config} and {\tt ,in} commands.  It
    moves to or executes a command in the current package's ``package
    for syntax,'' which is the package in which the forms $f$ in
    {\tt (define-syntax \cvar{name} \cvar{f})} are evaluated.

  \item \code{,new-package [\cvar{struct-name} \ldots]} \\
    The {\tt,new-package} command creates a new package and moves the
    command processor to it.  With no arguments, only the standard
    Scheme bindings are visible in the new package.  Otherwise, the
    structures specified as command arguments (and not the
    \texttt{scheme} structure) are opened in the new package.

\item \code{,structure \cvar{name} \cvar{interface}} \\
    The {\tt ,structure} command defines \cvar{name} in the
    configuration package to be a structure with interface
    \cvar{interface} based on the current package.

\end{description}



\section{Configuration packages}
\label{config-packages}

It is possible to set up multiple configuration packages.  The default
configuration package opens the following structures:
\begin{itemize}
\item {\tt module-system}, which exports {\tt define-structure} and the
      other configuration language keywords, as well as standard types
      and type constructors ({\tt :syntax}, {\tt :value}, {\tt proc}, etc.).
\item {\tt built-in-structures}, which exports structures that are
      built into the initial \hack{} image; these include
      {\tt scheme}, {\tt threads}, {\tt tables}, and {\tt records}.
\item {\tt more-structures}, which exports additional structures that
      are available in the development environment. 
      A complete listing
      can be found in the definition of \code{more-structures-interface}
      at the end of the file \code{scheme/packages.scm}.
\end{itemize}
Note that it does not open {\tt scheme}.

You can define additional configuration packages by making a package
 that opens {\tt module-\ok{}system} and, optionally,
 {\tt built-in-\ok{}structures},
 {\tt more-\ok{}structures}, or other structures that
 export structures and interfaces.

For example:
\begin{example}
> ,config (define-structure foo (export)
            (open module-system
                  built-in-structures
                  more-structures))
> ,in foo
foo> (define-structure x (export a b)
       (open scheme)
       (files x))
foo> 
\end{example}

Unfortunately, the above example does not work.
The problem is that every environment in which
 \code{define-structure} is used must also have a way to
 create ``syntactic towers''.
A new syntactic tower is required whenever a new environment is created for
 compiling the source code in the package associated with a new structure.
The environment's tower is used at compile time for
 evaluating the \cvar{macro-source} in
\begin{example}
(define-syntax \cvar{name} \cvar{macro-source})
(let-syntax ((\cvar{name} \cvar{macro-source}) \ldots) \cvar{body})
\end{example}
 and so forth.
It is a ``tower'' because that environment, in turn, has to say what environment
 to use if \code{macro-source} itself contains a use of \code{let-syntax}.

The simplest way to provide a tower maker is to pass on the one used by
 an existing configuration package.
The special form \code{export-syntactic-tower-maker} creates an interface
 that exports a configuration package's tower.
The following example uses \code{export-syntactic-tower-maker} and
 the \code{,structure} command to obtain a tower maker and create a new
 configuration environment.

\begin{example}
> ,config ,structure t (export-syntactic-tower-maker)
> ,config (define-structure foo (export)
            (open module-system
                  t
                  built-in-structures
                  more-structures))
\end{example}

Before \hack{}~1.9, \code{export-syntactic-tower-maker} was named
 \code{export-reflective-tower-maker}; this name is still supported for
 backwards compatibility.

%Talk about bootstrapping, define-all-operators, and usual-transforms?

\section{Discussion}

This module system was not designed as the be-all and end-all of
Scheme module systems; it was only intended to help us
organize the \hack{} system.  Not only does the module system
help avoid name clashes by keeping different subsystems in different
namespaces, it has also helped us to tighten up and generalize
\hack{}'s internal interfaces.  \hack{} is unusual among Lisp
implementations in admitting many different possible modes of
operation.  Examples of such multiple modes include the following:
\begin{itemize}
    \item Linking can be either static or dynamic.

    \item The development environment (compiler, debugger, and command
    processor) can run either in the same address space as the program
    being developed or in a different address space.  The environment and
    user program may even run on different processors under different
    operating systems\cite{Rees-Donald:Program}.

    \item The virtual machine can be supported by either
    of two implementations of its implementation language, Prescheme.
\end{itemize}
The module system has been helpful in organizing these multiple modes.
By forcing us to write down interfaces and module dependencies, the
module system helps us to keep the system clean, or at least to keep
us honest about how clean or not it is.

The need to make structures and interfaces second-class instead of
first-class results from the requirements of static program analysis:
it must be possible for the compiler and linker to expand macros and
resolve variable bindings before the program is executed.  Structures
could be made first-class (as in FX\cite{Sheldon-Gifford:Static}) if a
type system were added to Scheme and the definitions of exported
macros were defined in interfaces instead of in module bodies, but
even in that case types and interfaces would remain second-class.

The prohibition on assignment to imported bindings makes substitution
a valid optimization when a module is compiled as a block.  The block
compiler first scans the entire module body, noting which variables
are assigned.  Those that aren't assigned (only {\tt define}d) may be
assumed never assigned, even if they are exported.  The optimizer can
then perform a very simple-minded analysis to determine automatically
that some procedures can and should have their calls compiled in line.

The programming style encouraged by the module system is consistent
with the unextended Scheme language.  Because module system features
do not generally show up within module bodies, an individual module
may be understood by someone who is not familiar with the module
system.  This is a great aid to code presentation and portability.  If
a few simple conditions are met (no name conflicts between packages,
and use of {\tt files} in preference to
{\tt begin}), then a multi-module program can be loaded into a Scheme
implementation that does not support the module system.  The \hack{}
static linker satisfies these conditions, and can therefore run in
other Scheme implementations.  \hack{}'s bootstrap process, which is
based on the static linker, is therefore nonincestuous.  This
contrasts with most other integrated programming environments, such as
Smalltalk-80, where the system can only be built using an existing
version of the system itself.

Like ML modules, but unlike Scheme Xerox modules, this module system
is compositional.  That is, structures are constructed by single
syntactic units that compose existing structures with a body of code.
In Scheme Xerox, the set of modules that can contribute to an
interface is open-ended --- any module can contribute bindings to any
interface whose name is in scope.  The module system implementation is
a cross-bar that channels definitions from modules to interfaces.  The
module system described here has simpler semantics and makes
dependencies easier to trace.  It also allows for higher-order
modules, which Scheme Xerox considers unimportant.

%[Discuss use of module system in the \hack{} implementation?  Maybe
%give an extended excerpt from \hack{}'s configuration files?]
%
%[Discuss OPTIMIZE clause.]
%
%[Future work: ideas for anonymous structures and more of a module
%calculus; dealing with name conflicts; interface subtraction.]

%%% Local Variables: 
%%% mode: latex
%%% TeX-master: "manual"
%%% End: 
