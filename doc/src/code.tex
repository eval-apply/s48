% Latex Macros for Lisp code in text.
% Based on macros found in C. Rich's library.

\makeatletter

% \vobeyspaces turns all spaces into non-breakable spaces.
% Note: this is like \@vobeyspaces except without spurious space in defn.

{\catcode`\ =\active\gdef\vobeyspaces{\catcode`\ =\active\let =\@xobeysp}}

% \def\vobeytabs turns all tabs into 8 non-breakable spaces

{\catcode`\^^I=\active\gdef\vobeytabs{\catcode`\^^I=\active\let^^I=\xvobeytabs}}

\def\xvobeytabs{\@xobeysp\@xobeysp\@xobeysp\@xobeysp\@xobeysp\@xobeysp\@xobeysp\@xobeysp}

% \vobeylines turns all cr's into non-breakable \par's

{\catcode`\^^M=\active\gdef\vobeylines{\catcode`\^^M=\active\let^^M=\xvobeylines}}

\def\xvobeylines{\par\penalty10000}

% \obeycrsp turns cr's into non-breakable spaces

{\catcode`\^^M=\active\gdef\obeycrsp{\catcode`\^^M=\active\let^^M=\@xobeysp}}

%% \@noligs prevents ?` and !` from being treated as ligatures
%% added 19 April 86  [copied from Latex sources]

\begingroup
\catcode``=13
\gdef\@noligs{\let`=\@lquote}
\endgroup

% Set up code environment, in which most of the common special characters
% appearing in code are treated verbatim, namely: _ # & ^ $ ~ @ " %
%  *** JAR NEEDED $ AND _ IN SOME CODE ***

% Note: \ { } are still enabled so that macros can be called in this
% environment.  Use \\, \{ and \} to use these characters verbatim
% in this environment.  

% Note: this environment allows no breaking of lines whatsoever; not
% at spaces or hypens.  To arrange for a break use the standard \- macro,
% or the \= macro which breaks, but inserts nothing.  This is useful,
% for example for allowing hypenated identifiers to be broken, e.g.
% FOO-\=BAR.

\def\setupcode{\parsep=0pt\parindent=0pt
  \tt\frenchspacing\catcode``=13\@noligs%
  \def\\{\char`\\}%
  \@makeother\#\@makeother\&\@makeother\^%\@makeother\_\@makeother\$%
  \@makeother\`\@makeother\'%
  \@makeother\~\@makeother\@\@makeother\"\@makeother\%\vobeytabs\vobeyspaces}

% Code environment as described above.  Note that blank lines are
% not preserved, and lines are not kept on one page.  Code is
% indented by the same amount as quotes.
% Note: to increase left margin, use \leftmargini=1in.
%  was  {\list{}{\parsep=0pt}\item[]\setupcode\obeylines}%
% then {\list{\parsep=0pt\listparindent=0pt\leftmargin=0pt}{}\item[]\setupcode%

\newenvironment{bigcode}%
  {\list{}{\parsep=0pt\leftmargin=0pt\labelwidth=0pt\itemindent=0pt%
\listparindent=0pt}\item[]\setupcode%
\obeylines}%
  {\endlist}

% Code is just like bigcode, but everything inside is kept on one page
% Note: This actually works by setting a huge penalty for breaking
% between lines of code.
%  was  {\list{}{\parsep=0pt}\item[]\setupcode\vobeylines}%

\newenvironment{code}%
  {\list{}{\parsep=0pt\leftmargin=0pt\labelwidth=0pt\itemindent=0pt%
\listparindent=0pt}\item[]\setupcode%
\vobeylines}%
  {\endlist}

% Reasonable separation between lines of code

\newcommand{\codeskip}{\penalty0\vspace{2ex}}

% \cd is used to build a code environment in the middle of text.
% Note: only difference from display code is that cr's are taken
% as unbreakable spaces instead of \par's.

\newcommand{\cd}{\begingroup\setupcode\obeycrsp\startcode}

\newcommand{\startcode}[1]{#1\endgroup}

%\setbox0\hbox{\@xobeysp}\hline{43\wd0}

\makeatother
