
\chapter{User's guide}

This chapter details Scheme~48's user interface: its command-line arguments,
 command processor, debugger, and so forth.

\section{Command line arguments}

A few command line arguments are processed by Scheme~48 as
 it starts up.

\code{scheme48}
[\code{-i} \cvar{image}]
[\code{-h} \cvar{heapsize}]
% [\code{-s} \cvar{stacksize}]
[\code{-a} \cvar{argument \ldots}]

\begin{description}
\item[{\tt -i} \cvar{image}]
    specifies a heap image file to resume.  This defaults to a heap
    image that runs a Scheme command processor.  Heap images are
    created by the \code{,dump} and \code{,build commands}, for which see below.

\item[{\tt -h} \cvar{heapsize}]
    specifies how much space should be reserved for allocation.
    \cvar{Heapsize} is in words (where one word = 4 bytes), and covers both
    semispaces, only one of which is in use at any given time (except
    during garbage collection).  Cons cells are currently 3 words, so
    if you want to make sure you can allocate a million cons cells,
    you should specify \code{-h 6000000} (actually somewhat more than this,
    to account for the initial heap image and breathing room).
    The default heap size is 3000000 words.  The system will use a
    larger heap if the specified (or default) size is less than
    the size of the image being resumed.
% #### true only for twospace GC; with BIBOP, specifies size of whole heap
%      (or unlimited heap size with -h 0)

%\item[{\tt -s} \cvar{stacksize}]
%    specifies how much space should be reserved for the continuation
%    and environment stack.  If this space is exhausted, continuations
%    and environments are copied to the heap.  \cvar{Stacksize} is in words
%    and defaults to 2500.

\item[{\tt -a} \cvar{argument \ldots}]
    is only useful with images built using \code{,build}.
    The arguments are passed as a list of OS strings (see section
    \ref{os-strings}) to the procedure specified
    in the \code{,build} command. For example:
\begin{example}
> ,open os-strings
> (define (f xs)
    (write (map os-string->string xs))
    (newline)
    0)                                        ;must return an integer
> ,build f foo.image
> ,exit
\% scheme48vm -i foo.image -a mumble "foo x" -h 5000000
("mumble" "foo x" "-h" "5000000")
\%
\end{example}

\item[{\tt -I} \cvar{image} \cvar{argument \ldots}]
    is equivalent to {\tt % -h 0
      -i \cvar{image} -a \cvar{argument \ldots}}.
    On most Unix-like systems, a heap image can be made executable with the
    following Bourne shell commands:
\begin{example}
\% (echo '\#!\cvar{/s48/install/prefix}/lib/scheme48-1.9t/scheme48vm -I'
   cat \cvar{original.image}) >\cvar{new.image}
\% chmod +x \cvar{new.image}
\end{example}
\end{description}

The usual definition of the \code{s48} or \code{scheme48} command is actually a
 shell script that starts up the Scheme~48 virtual machine with a
 \code{-i \cvar{imagefile}}
specifying the development environment heap image and a
 \code{-o \cvar{vm-executable}} specifying the location of the virtual-machine
 executable (the executable is needed for loading external code on some
 versions of Unix; see section~\ref{dynamic-externals} for more information).
The file \code{go} in the Scheme~48 installation source directory is an example
 of such a shell script.

\section{Command processor}

When you invoke the default heap image, a command processor starts
 running.
The command processor acts as both a read-eval-print loop, reading
 expressions, evaluating them, and printing the results, and as
 an interactive debugger and data inspector.
See Chapter~\ref{chapter:command-processor} for
a description of the command processor.

\section{Editing}

We recommend running Scheme~48 under GNU Emacs or XEmacs using the
 \code{cmuscheme48} command package.
This is in the Scheme~48 distribution's \code{emacs/} subdirectory and
 is included in XEmacs's \code{scheme} package.
It is a variant of the \code{cmuscheme} library, which
 comes to us courtesy of Olin Shivers, formerly of CMU.
You might want to put the following in your Emacs init file (\code{.emacs}):
\begin{example}
(setq scheme-program-name "scheme48")
(autoload 'run-scheme
          "cmuscheme48"
          "Run an inferior Scheme process."
          t)
\end{example}
The Emacs function \code{run-scheme} can then be used to start a process
 running the program \code{scheme48} in a new buffer.
To make the \code{autoload} and \code{(require \ldots)} forms work, you will
also need
to put the directory containing \code{cmuscheme} and related files in your
emacs load-path:
\begin{example}
(setq load-path
  (append load-path '("\cvar{scheme-48-directory}/emacs")))
\end{example}
Further documentation can be found in the files \code{emacs/cmuscheme48.el} and
\code{emacs/comint.el}.

\section{Performance}
\label{section:performance}

If you want to generally have your code run faster than it normally
would, enter \code{inline-values} mode before loading anything.  Otherwise
calls to primitives (like \code{+} and \code{cons}) and in-line procedures
(like \code{not} and \code{cadr}) won't be open-coded, and programs will run
more slowly.

The system doesn't start in \code{inline-values} mode by default because the
Scheme report permits redefinitions of built-in procedures.  With
this mode set, such redefinitions don't work according to the report,
because previously compiled calls may have in-lined the old
definition, leaving no opportunity to call the new definition.

\code{Inline-values} mode is controlled by the \code{inline-values} switch.
\code{,set inline-values} and \code{,unset inline-values} turn it on and off.

\section{Disassembler}

The \code{,dis} command prints out the disassembled byte codes of a procedure.
\begin{example}
> ,dis cons
cons
  0 (protocol 2)
  2 (pop)
  3 (make-stored-object 2 pair)
  6 (return)
> 
\end{example}
The current byte codes are listed in the file \code{scheme/vm/interp/arch.scm}.
A somewhat out-of-date description of them can be found in
\cite{Kelsey-Rees:Scheme48}.

The command argument is optional; if unsupplied it defaults to the
current focus object (\code{\#\#}).

The disassembler can also be invoked on continuations and templates.

\section{Module system}
\label{module-guide}

This section gives a brief description of modules and related entities.
For detailed information, including a description of the module
 configuration language, see 
chapter \ref{chapter:modules}.

% JAR says: this paragraph is muddy.

A {\em module} is an isolated namespace, with visibility of bindings
 controlled by module descriptions written in a special
 configuration language.
A module may be instantiated as a {\em package}, which is an environment
 in which code can be evaluated.
Most modules are instantiated only once and so have a unique package.
A {\em structure} is a subset of the bindings in a package.
Only by being included in a structure can a binding be
 made visible in other packages.
A structure has two parts, the package whose bindings are being exported
 and the set of names that are to be exported.
This set of names is called an {\em interface}.
A module then has three parts:
\begin{itemize}
\item a set of structures whose bindings are to be visible within the module
\item the source code to be evaluated within the module
\item a set of exported interfaces
\end{itemize}
Instantiating a module produces a package and a set of structures, one for
 each of the exported interfaces.

The following example uses \code{define-structure} to create a module that
 implements simple cells as pairs, instantiates this module, and binds the
 resulting structure to \code{cells}.
The syntax \code{(export \cvar{name \ldots})} creates an interface
 containing \cvar{name \ldots}.
The \code{open} clause lists structures whose bindings are visible
 within the module.
The \code{begin} clause contains source code.
\begin{example}
(define-structure cells (export make-cell
                                cell-ref
                                cell-set!)
  (open scheme)
  (begin (define (make-cell x)
           (cons 'cell x))
         (define cell-ref cdr)
         (define cell-set! set-cdr!)))
\end{example}

Cells could also have been implemented using the
record facility described in section~\ref{records}
 and available in structure \code{define-record-type}.
\begin{example}
(define-structure cells (export make-cell
                                cell-ref
                                cell-set!)
  (open scheme define-record-types)
  (begin (define-record-type cell :cell
           (make-cell value)
           cell?
           (value cell-ref cell-set!))))
\end{example}

With either definition the resulting structure can be used in other
 modules by including \code{cells} in an \code{open} clause.

The command interpreter is always operating within a particular package.
Initially this is a package in which only the standard Scheme bindings
 are visible.
The bindings of other structures can be made visible by using the 
\code{,open} command described in section~\ref{module-command-guide} below.

Note that this initial package does not include the configuration language.
Module code needs to be evaluated in the configuration package, which can
 be done by using the {\code ,config} command:
\begin{example}
> ,config (define-structure cells \ldots)
> ,open cells
> (make-cell 4)
'(cell . 4)
> (define c (make-cell 4))
> (cell-ref c)
4
\end{example}

\section{Library}

A number of useful utilities are either built in to Scheme~48 or can
be loaded from an external library.  These utilities are not visible
in the user environment by default, but can be made available with the
\code{open} command.  For example, to use the \code{tables} structure, do
\begin{example}
> ,open tables
> 
\end{example}

If the utility is not already loaded, then the \code{,open} command will
 load it.
Or, you can load something explicitly (without opening it) using the
\code{load-package} command:
\begin{example}
> ,load-package queues
> ,open queues
\end{example}

When loading a utility, the message "Note: optional optimizer not
invoked" is innocuous.  Feel free to ignore it.

See also the package system documentation, in
chapter~\ref{chapter:modules}.

Not all of the the libraries available in Scheme~48 are described in this
 manual.
All are listed in files \code{rts-packages.scm},
 \code{comp-packages.scm}, \code{env-packages.scm}, and
 \code{more-packages.scm} in the \code{scheme} directory of the distribution,
 and the bindings they
 export are listed in \code{interfaces.scm} and
 \code{more-interfaces.scm} in the same directory.

%architecture
%    Information about the virtual machine.  E.g.
%      (enum op eq?) => the integer opcode of the EQ? instruction
%
%big-scheme
%    Many generally useful features.  See doc/big-scheme.txt.
%
%bigbit
%    Extensions to the bitwise logical operators (exported by
%    the BITWISE structure) so that they operate on bignums.
%    To use these you should do
%
%        ,load-package bigbit
%        ,open bitwise
%
%conditions
%    Part of the condition system: DEFINE-CONDITION-PREDICATE and
%    routines for examining condition objects.  (See also handle,
%    signals.)
%
%defpackage
%    The module system: DEFINE-STRUCTURE and DEFINE-INTERFACE.
%
%destructuring
%    DESTRUCTURE macro.  See doc/big-scheme.txt.
%
%display-conditions
%    Displaying condition objects.
%        (DISPLAY-CONDITION condition port) \goesto{} unspecific
%          Display condition in an easily readable form.  E.g.
%\begin{example}
%          > ,open display-conditions handle conditions
%          > (display-condition
%             (call-with-current-continuation
%               (lambda (k)
%                 (with-handler (lambda (c punt)
%                                 (if (error? c)
%                                     (k c)
%                                     (punt)))
%                   (lambda () (+ 1 'a)))))
%             (current-output-port))
%
%          Error: exception
%                 (+ 1 'a)
%          > 
%\end{example}
%
%extended-ports
%    Ports for reading from and writing to strings, and related things.
%    See doc/big-scheme.txt.
%
%filenames
%    Rudimentary file name parsing and synthesis.  E.g.
%    file-name-directory and file-name-nondirectory are as in Gnu emacs.
%
%floatnums
%    Floating point numbers.  These are in a very crude state; use at
%    your own risk.  They are slow and do not read or print correctly.
%
%fluids
%    Dynamically bound "variables."
%      (MAKE-FLUID top-level-value) \goesto{} a "fluid" object
%      (FLUID fluid) \goesto{} current value of fluid object
%      (SET-FLUID! fluid value) \goesto{} unspecific; changes current value of
%        fluid object
%      (LET-FLUID fluid value thunk) \goesto{} whatever thunk returns
%        Within the dynamic extent of execution of (thunk), the fluid
%        object has value as its binding (unless changed by SET-FLUID!
%        or overridden by another LET-FLUID).
%    E.g.
%      (define f (make-fluid 7))
%      (define (baz) (+ (fluid f) 1))
%      (baz)   ;\goesto{} 8
%      (let-fluid f 4 (lambda () (+ (baz) 1)))  ;\goesto{} 6
%
%formats
%    A simple FORMAT procedure, similar to Common Lisp's or T's.
%    See doc/big-scheme.txt for documentation.
%
%handle
%    Part of the condition system.
%      (WITH-HANDLER handler thunk) \goesto{} whatever thunk returns.
%        handler is a procedure of two arguments.  The first argument
%        is a condition object, and the second is a "punt" procedure.
%        The handler should examine the condition object (using ERROR?,
%        etc. from the CONDITIONS structure).  If it decides not to do
%        anything special, it should tail-call the "punt" procedure.
%        Otherwise it should take appropriate action and perform a
%        non-local exit.  It should not just return unless it knows
%        damn well what it's doing; returns in certain situations can
%        cause VM crashes.
%
%interrupts
%    Interrupt system
%
%ports
%    A few extra port-related operations, notably FORCE-OUTPUT.
%
%pp
%    A pretty-printer.  (p \cvar{exp}) will pretty-print the result of \cvar{exp},
%    which must be an S-expression.  (Source code for procedures is not
%    retained or reconstructed.)  You can also do (p \cvar{exp} \cvar{port}) to
%    print to a specific port.
%
%    The procedure pretty-print takes three arguments: the object to be
%    printed, a port to write to, and the current horizontal cursor
%    position.  If you've just done a newline, then pass in zero for
%    the position argument.
%
%    The algorithm is very peculiar, and sometimes buggy.
%
%queues
%    FIFO queues.
%
%random
%    Not-very-random random number generator.  The \cvar{seed} should be between
%    0 and 2$^{28}$ exclusive.
%
%        > (define random (make-random \cvar{seed}))
%        > (random) \goesto{} pseudo-random number between 0 and 2$^{28}$
%
%receiving
%    Convenient interface to the call-with-values procedure, like
%    Common Lisp's multiple-value-bind macro.  See doc/big-scheme.txt.
%
%records
%    MAKE-RECORD-TYPE and friends.  See the Scheme of Things column in
%    Lisp Pointers, volume 4, number 1, for documentation.
%
%recnums
%    Complex numbers.  This should be loaded (e.g. with ,load-package)
%    but needn't be opened.
%
%search-trees
%    Balanced binary search trees.  See comments at top of
%    big/search-tree.scm. 
%
%signals
%    ERROR, WARN, and related procedures.
%
%sort
%    Online merge sort (see comment at top of file big/sort.scm).
%
%        (sort-list \cvar{list} \cvar{pred})
%        (sort-list! \cvar{list} \cvar{pred})
%
%sicp
%    Compatibility package for the Scheme dialect used in the book
%    "Structure and Interpretation of Computer Programs."
%
%sockets
%    Interface to Unix BSD sockets.  See comments at top of file
%    misc/socket.scm.
%
%threads
%    Multitasking.  See doc/threads.txt.
%
%util
%    SUBLIST, ANY, REDUCE, FILTER, and some other useful things.
%
%weak
%    Weak pointers and populations.
%        (MAKE-WEAK-POINTER thing) => weak-pointer
%        (WEAK-POINTER-REF weak-pointer) => thing or \code{\#f}
%          \code{\#f} if the thing has been gc'ed.
%
%writing
%        (RECURRING-WRITE thing port recur) => unspecific
%          This is the same as WRITE except that recursive calls invoke
%          the recur argument instead of WRITE.  For an example, see
%          the definition of LIMITED-WRITE in env/dispcond.scm, which
%          implements processing similar to common Lisp's *print-level*
%          and *print-length*.

%%% Local Variables: 
%%% mode: latex
%%% TeX-master: "manual"
%%% End: 
