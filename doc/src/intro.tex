
\chapter{Introduction}

Scheme~48 is an implementation of the Scheme programming language as
described in the Revised$^5$ Report on the Algorithmic Language
 Scheme~\cite{R5RS}.
It is based on a compiler and interpreter for a virtual Scheme
machine.  Scheme~48 tries to be faithful to the Revised$^5$ Scheme
Report, providing neither more nor less in the initial user
environment.  (This is not to say that more isn't available in other
environments; see below.)  Support for numbers is weak: bignums are
slow and floating point is almost nonexistent (see description of
floatnums, below).

% JAR says: replace zurich with mumble.net or ...

Scheme~48 is under continual development.
Please report bugs, especially in the VM, especially core dumps, to
scheme-48-bugs@zurich.ai.mit.edu.  Include the version number x.yy
from the "Welcome to Scheme~48 x.yy" greeting message in your bug
report.  It is a goal of this project to produce a bullet-proof
system; we want no bugs and, especially, no crashes.  (There are a few
known bugs, listed in the {\tt doc/todo.txt} file that comes with the
distribution.)

Send mail to scheme-48-request@zurich.ai.mit.edu to be put on a
mailing list for announcements, discussion, bug reports, and bug
fixes.

The name `Scheme~48' commemorates our having written the original version
 in forty-eight hours, on August 6th and 7th, 1986.

