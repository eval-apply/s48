\chapter{Command processor}
\label{chapter:command-processor}

This chapter details Scheme~48's command processor, which incorporates
 both a read-eval-print loop and an interactive debugger.
At the \code{>} prompt, you can type either a Scheme form
 (expression or definition) or a command beginning with a comma.
In
\link*{inspection mode}[inspection mode (see section~\Ref)]{inspector}
 the prompt changes to \code{:} and commands
 no longer need to be preceded by a comma; input beginning with
 a letter or digit is assumed to be a command, not an expression.
In inspection mode the command processor prints out a
 menu of selectable components for the current object of interest.

\section{Current focus value and {\tt \#\#}}

The command processor keeps track of a current {\em focus value}.
This value is normally the last value returned by a command.
If a command returns multiple values the focus object is a list of the
 values.
The focus value is not changed if a command returns no values or 
 a distinguished `unspecific' value.
Examples of forms that return this unspecific value are definitions,
 uses of \code{set!}, and \code{(if \#f 0)}.
It prints as \code{\#\{Unspecific\}}.

The reader used by the command processor reads \code{\#\#} as a special
 expression that evaluates to the current focus object.
\begin{example}
> (list 'a 'b)
'(a b)
> (car ##)
'a
> (symbol->string ##)
"a"
> (if #f 0)
#\{Unspecific\}
> ##
"a"
> 
\end{example}

\section{Command levels}

If an error, keyboard interrupt, or other breakpoint occurs, or the
 \code{,push} command is used, the command
 processor invokes a recursive copy of itself, preserving the dynamic state of
 the program when the breakpoint occured.
The recursive invocation creates a new {\em command level}.
The command levels form a stack with the current level at the top.
The command prompt indicates the number of stopped levels below the
 current one: \code{>} or \code{:} for the
 base level and \code{\cvar{n}>} or \code{\cvar{n}:} for all other levels,
 where \cvar{n} is the command-level nesting depth.
The \code{auto-levels} switch
 described below can be used to disable the automatic pushing of new levels.

The command processor's evaluation package and the value of the
 \code{inspect-focus-value} switch are local to each command level.
They are preserved when a new level is pushed and restored when
 it is discarded.
The settings of all other switches are shared by all command levels.

\begin{description}
\item $\langle{}$eof$\rangle{}$\\
    Discards the current command level and resumes running the level down.
    $\langle{}$eof$\rangle{}$ is usually
    control-\code{D} at a Unix shell or control-\code{C} control-\code{D} using
    the Emacs \code{cmuscheme48} library.

\item \code{,pop}\\
 The same as $\langle{}$eof$\rangle{}$.

\item \code{,proceed [\cvar{exp} \ldots}]\\
 Proceed after an interrupt or error, resuming the next command
 level down, delivering the values of \cvar{exp~\ldots} to the continuation.
 Interrupt continuations discard any returned values.
 \code{,Pop} and \code{,proceed} have the same effect after an interrupt
 but behave differently after errors.
 \code{,Proceed} restarts the erroneous computation from the point where the
 error occurred (although not all errors are proceedable) while
 \code{,pop} (and $\langle{}$eof$\rangle{}$) discards it and prompts for
 a new command.

\item \code{,push}\\
 Pushes a new command level on above the current one.
 This is useful if the \code{auto-levels} switch has been used
 to disable the automatic pushing of new levels for errors and interrupts.

\item \code{,reset [\cvar{number}]}\\
  Pops down to a given level and restarts that level.
  \cvar{Number} defaults to zero, \code{,reset} restarts the command
  processor, discarding all existing levels.

\end{description}

Whenever moving to an existing level, either by sending
 an $\langle{}$eof$\rangle{}$
 or by using \code{,reset} or the other commands listed above,
 the command processor runs all of the \code{dynamic-wind} ``after'' thunks
 belonging to stopped computations on the discarded level(s).

\section{Logistical commands}
\begin{description}
\item \code{,load \cvar{filename \ldots}}\\
    Loads the named Scheme source file(s).
    Easier to type than \code{(load "\cvar{filename}")} because you don't have to
    shift to type the parentheses or quote marks.  Also, it works in
    any package, unlike \code{(load "\cvar{filename}")}, which will work only
    work in packages in which the variable \code{load} is defined appropriately.

\item \code{,exit [\cvar{exp}]}
    Exits back out to shell (or executive or whatever invoked Scheme~48
    in the first place).
    \cvar{Exp} should evaluate to an integer.  The
    integer is returned to the calling program.
    The default value of \cvar{exp} is zero, which, on Unix,
    is generally interpreted as success.
\end{description}

\section{Module commands}
\label{module-command-guide}

There are many commands related to modules.
Only the most commonly used module commands are described here;
 documentation for the
 rest can be found in \link*{the module chapter}[section~\Ref]{module-commands}.
There is also
\link*{a brief description of modules, structures, and packages}
[a brief description of modules, structures, and packages in section~\Ref]
{module-guide} below.

\begin{description}
\item \code{,open \cvar{structure \ldots}}\\
Makes the bindings in the \cvar{structure}s visible in the current package.
The packages associated with the \cvar{structure}s will be loaded if
 this has not already been done (the \code{ask-before-loading} switch
 can be used disable the automatic loading of packages).

\item \code{,config [\cvar{command}]}\\
Executes \cvar{command} in the \code{config} package, which includes
 the module configuration language.
For example, use
\begin{example}
,config ,load \cvar{filename}
\end{example}
to load a file containing module definitions.
If no \cvar{command} is given, the \code{config} package becomes the
 execution package for future commands.

\end{description}

\section{Debugging commands}
\label{debug-commands}

\begin{description}
\item \code{,preview}\\
    Somewhat like a backtrace, but because of tail recursion you see
    less than you might in debuggers for some other languages.
    The stack to display is chosen as follows:
\begin{enumerate}
    \item If the current focus object is a continuation or a thread,
       then that continuation or thread's stack is displayed.
    \item Otherwise, if the current command level was initiated because of
       a breakpoint in the next level down, then the stack at that
       breakpoint is displayed.
    \item Otherwise, there is no stack to display and a message is printed
       to that effect.
\end{enumerate}
    One line is printed out for each continuation on the chosen stack,
    going from top to bottom.

\item \code{,run \cvar{exp}}\\
    Evaluates \cvar{exp}, printing the result(s) and making them
    (or a list of them, if \cvar{exp} returns multiple results)
    the new focus object.
    The \code{,run} command is useful when writing
    \link*{command programs}
    [command programs, which are described in section~\Ref{} below]
    {command-programs}.

\item \code{,trace \cvar{name} \ldots}\\
    Start tracing calls to the named procedure or procedures.
    With no arguments, displays all procedures currently traced.
    This affects the binding of \cvar{name}, not the behavior of the
    procedure that is its current value.  \cvar{Name} is redefined
    to be a procedure that prints a message,
    calls the original value of \cvar{name}, prints another
    message, and finally passes along the value(s) returned by the
    original procedure.

\item \code{,untrace \cvar{name} \ldots}\\
    Stop tracing calls to the named procedure or procedures.
    With no argument, stop tracing all calls to all procedures.

\item \code{,condition}\\
    The \code{,condition} command displays the condition object
    describing the error or interrupt that initiated the current
    command level.  The condition object becomes the current focus
    value.  This is particularly useful in conjunction with
    the inspector.  For example, if a procedure is passed the wrong number of
    arguments, do \code{,condition} followed by
     \code{,inspect} to inspect the
    procedure and its arguments.

\item \code{,bound?\ \cvar{name}}\\
    Display the binding of \cvar{name}, if there is one, and otherwise
    prints `\code{Not bound}'.
 
\item \code{,expand \cvar{form}}
\T\vspace{-1em}
\item \code{,expand-all \cvar{form}}\\
    Show macro expansion of \cvar{form}, if any.
    \code{,expand} performs a single macro expansion while
    \code{,expand-all} fully expands all macros in \cvar{form}.

\item \code{,where \cvar{procedure}}\\
    Display name of file containing \cvar{procedure}'s source code.
\end{description}

\section{Switches}

There are a number of binary switches that control the behavior of the
 command processor.
They can be set using the \code{,set} and \code{,unset} commands.

\begin{description}
\item \code{,set \cvar{switch} [on | off | ?]}\\
    This sets the value of mode-switch \cvar{switch}.
    The second argument defaults to \code{on}.
    If the second argument is \code{?} the value of \cvar{switch} is
    is displayed and not changed.
    Doing \code{,switch ?} will display a list of the switches and
    their current values.

\item \code{,unset \cvar{switch}}\\
    \code{,unset \cvar{switch}} is the same as
    \code{,set \cvar{switch} off}.
\end{description}

The switches are as follows:
\begin{description}
\item \code{batch}\\
    In `batch mode' any error or interrupt that comes up will cause
    Scheme~48 to exit immediately with a non-zero exit status.  Also,
    the command processor doesn't print prompts.  Batch mode is
    off by default.

\item \code{auto-levels}\\
    Enables or disables the automatic pushing of a new command level when
    an error, interrupt, or other breakpoint occurs.
    When enabled (the default), breakpoints push a new command level,
    and $\langle{}$eof$\rangle{}$ (see above)
    or \code{,reset} is required to return to top level.  The effects of
    pushed command levels include:
\begin{itemize}
\item a longer prompt
\item retention of the continuation in effect at the point of errors
\item confusion among some newcomers
\end{itemize}
    With \code{auto-levels} disabled one must issue a
    \code{,push} command immediately
    following an error in order to retain the error continuation for
    debugging purposes; otherwise the continuation is lost after the
    next evaluation request.  If you don't know anything about the
    available debugging tools, then levels might as well be disabled.

\item \code{inspect-focus-value}\\
    Enable or disable `inspection' mode, which is used for inspecting
    data structures and continuations.
    \link*{Inspection mode is desribed below}
    [Inspection mode is described in section~\Ref]
    {inspector}.

\item \code{break-on-warnings}\\
    Enter a new command level when a warning is produced, just as
    when an error occurs.  Normally warnings only result in a displayed
    message and the program does not stop executing.

\item \code{ask-before-loading} \\
    If on, the system will ask before loading modules that are arguments
    to the \code{,open} command.  \code{Ask-before-loading} is off by
    default.
\begin{example}
> ,set ask-before-loading
will ask before loading modules
> ,open random
Load structure random (y/n)? y
>
\end{example}

\item \code{load-noisily}\\
    When on, the system will print out the names of modules and files
    as they are loaded.  \code{load-noisily} is off by default.
\begin{example}
> ,set load-noisily
will notify when loading modules and files
> ,open random
[random /usr/local/lib/scheme48/big/random.scm]
> 
\end{example}

\item \code{inline-values}\\
This controls whether or not the compiler is allowed to substitute
 variables' values in-line.
When \code{inline-values} mode is on,
some Scheme procedures will be substituted in-line; when it is off,
none will.
\link*{The performance section}[Section~\Ref]{section:performance}
has more information.

\end{description}

\section{Inspection mode}
\label{inspector}

There is a data inspector available via the \code{,inspect} and
 \code{,debug} commands or by setting the \code{inspect-focus-value} switch.
The command processor can be taken out of inspection mode by
 using the \code{q} command, by unsetting the \code{inspect-focus-value} switch,
 or by going to a command level where the the \code{inspect-focus-value} is not
 set.
When in inspection mode input that begins with
  a letter or digit is read as a command, not as an expression.
To see the value of a variable or number, do \code{(begin \cvar{exp})}
  or use the \code{,run \var{exp}} command.

In inspection mode the command processor prints out a
 menu of selectable components for the current focus object.
To inspect a particular component, just type the corresponding number in
 the menu.
That component becomes the new focus object.
For example:
\begin{example}
> ,inspect '(a (b c) d)
(a (b c) d)

[0] a
[1] (b c)
[2] d
: 1
(b c)

[0] b
[1] c
: 
\end{example}

When a new focus object is selected the previous one is pushed onto a
 stack.
You can pop the stack, reverting to the previous object, with
 the \code{u} command, or use the \code{stack} command to move to
 an earlier object.

%\begin{description}
%\item \code{stack}\\
%    Prints the current stack out as a menu.
%    Selecting an item pops all higher values off of the stack and
%    makes that item the current focus value.
%\end{description}
%

Commands useful when in inspection mode:
\begin{itemize}
\item\code{u} (up) pop object stack
\item\code{m} (more) print more of a long menu
\item\code{(\ldots)} evaluate a form and select result
\item\code{q} quit
\item\code{template} select a closure or continuation's template
 (Templates are the static components of procedures; these are found
  inside of procedures and continuations, and contain the quoted
  constants and top-level variables referred to by byte-compiled code.)
\item\code{d} (down) move to the next continuation
 (current object must be a continuation)
\item\code{menu} print the selection menu for the focus object
\end{itemize}

Multiple selection commands (\code{u}, \code{d}, and menu indexes)
 may be put on a single line.

%\code{\#\#} is always the object currently being inspected.
%After a \code{q}
%command,
%or an error in the inspector, \code{\#\#} is the last object that was being
%inspected.

Inspection mode is particularly useful with procedures, continuations,
and records.

All ordinary commands are available when in inspection mode.
Similarly, the inspection commands can be used when not in inspection
 mode.
For example:
\begin{example}
> (list 'a '(b c) 'd)
'(a (b c) d)
> ,0
'(b c)
> ,menu
[0] b
[1] c
> 
\end{example}

If the current command level was initiated because of
 a breakpoint in the next level down, then 
 \code{,debug} will invoke the inspector on the
 continuation at the point of the error.
The \code{u} and \code{d} (up and down)
commands then make the inspected-value stack look like a conventional stack
debugger, with continuations playing the role of stack frames.  \code{D} goes
to older or deeper continuations (frames), and \code{u} goes back up to more
recent ones.

\section{Command programs}
\label{command-programs}

The \code{exec} package contains procedures that are used
 to execute the command processor's commands.
A command \code{,\cvar{foo}} is executed by applying the value of
 the identifier \cvar{foo} in the \code{exec} package to
 the (suitably parsed) command arguments.

\begin{description}
\item \code{,exec [\cvar{command}]}
   Evaluate \cvar{command} in the \code{exec} package.
   For example, use
\begin{example}
,exec ,load \cvar{filename}
\end{example}
to load a file containing commands.
If no \cvar{command} is given, the \code{exec} package becomes the
 execution package for future commands.
\end{description}

The required argument types are as follows:
\begin{itemize}
\item filenames should be strings
\item other names and identifiers should be symbols
\item expressions should be s-expressions
\item commands (as for \code{,config} and \code{,exec} itself)
 should be lists of the form \code(\cvar{command-name} \cvar{argument} ...)
 where \cvar{command-name} is a symbol.
\end{itemize}

For example, the following two commands are equivalent:
\begin{example}
,config ,load my-file.scm

,exec (config '(load "my-file.scm"))
\end{example}

The file \code{scheme/vm/load-vm.scm} in the source directory contains an
 example of an \code{exec} program.

\section{Building images}

\begin{description}
\item \code{,dump \cvar{filename} [\cvar{identification}]}\\
    Writes the current heap out to a file, which can then be run using the
    virtual machine.  The new image file includes the command processor.
    If present, \cvar{identification}
    should be a string (written with double quotes); this string will
    be part of the greeting message as the image starts up.

\item \code{,build \cvar{exp} \cvar{filename}}\\
    Like \code{,dump}, except that the image file contains the value of
    \cvar{exp}, which should be a procedure of one argument, instead of
    the command processor.  When
    \cvar{filename} is resumed, that procedure will be invoked on the VM's
    \code{-a} arguments, which are passed as a list of strings.  The
    procedure should return an integer which is
    returned to the program that invoked the VM.  The command
    processor and debugging system are not included in the image
    (unless you go to some effort to preserve them, such as retaining
    a continuation).

    Doing \code{,flush} before building an image will reduce the amount
    of debugging information in the image, making for a smaller
    image file, but if an error occurs, the error message may be less
    helpful.  Doing \code{,flush source maps} before loading any programs
    used in the image will make it still smaller.
    See \link*{the description of \code{flush}}[section~\Ref]{resource-commands}
    for more information.

\end{description}
    
\section{Resource query and control}
\label{resource-commands}.

\begin{description}
\item \code{,time \cvar{exp}}\\
    Measure execution time.

\item \code{,collect}\\
    Invoke the garbage collector.  Ordinarily this happens
    automatically, but the command tells how much space is available
    before and after the collection.

\item \code{,keep \cvar{kind}}
\T\vspace{-1em}
\item \code{,flush \cvar{kind}}\\
    These control the amount of debugging information retained after
    compiling procedures.  This information can consume a fair amount
    of space.  \cvar{kind} is one of the following:
\begin{itemize}
\item \code{maps} - environment maps (local variable names, for inspector)
\item \code{source} - source code for continuations (displayed by inspector)
\item \code{names} - procedure names (as displayed by \code{write} and in error
       messages)
\item \code{files}  - source file names
\end{itemize}
    These commands refer to future compilations only, not to procedures
    that already exist.  To have any effect, they must be done before
    programs are loaded.  The default is to keep all four types.

\item \code{,flush}\\
    The flush command with no argument deletes the database of names
    of initial procedures.  Doing \code{,flush} before a \code{,build} or
    \code{,dump}
    will make the resulting image significantly smaller, but will
    compromise the information content of many error
    messages.
\end{description}

\section{Threads}

Each command level has its own set of threads.  These threads are suspended
 when a new level is entered and resumed when the owning level again becomes
 the current level.
A thread that raises an error is not resumed unless
 explicitly restarted using the \code{,proceed} command.
In addition to any threads spawned by the user, each level has a thread
 that runs the command processor on that level.
A new command-processor thread is started if the current one
 dies or is terminated.
When a command level is abandoned for a lower level, or when
 a level is restarted using \code{,reset}, all of the threads on that
 level are terminated and any \code{dynamic-wind} ``after'' thunks are run.

The following commands are useful when debugging multithreaded programs:
\begin{description}
\item \code{,resume [\cvar{number}]}\\
  Pops out to a given level and resumes running all threads at that level.
  \cvar{Number} defaults to zero.

\item \code{,threads}\\
    Invokes the inspector on a list of the threads running at the
    next lower command level.

\item \code{,exit-when-done [\cvar{exp}]}\\
    Waits until all user threads have completed and then
    exits back out to shell (or executive or whatever invoked Scheme~48
    in the first place).
    \cvar{Exp} should evaluate to an integer which is then
    returned to the calling program.

%\item \code{,spawn \cvar{exp} [\cvar{name}]}\\
%    Starts a new thread running \cvar{exp} on next command level down.
%    The optional \cvar{name} is used for printing and debugging.
%
%\item \code{,suspend [\cvar{exp}]}
%\T\vspace{-1em}
%\item \code{,continue [\cvar{exp}]}
%\T\vspace{-1em}
%\item \code{,kill [\cvar{exp}]}\\
%    Suspend, unsuspend, and terminate a thread, respectively.
%    Suspended threads are not run until unsuspended, terminated
%    threads are never run again.
%    \cvar{Exp} should evaluate to a thread.
%    If \cvar{exp} is not present, the current focus object is used.
%

% example of ,threads ,suspend ...
\end{description}

\section{Quite obscure}

\begin{description}
\item \code{,go \cvar{exp}}\\
    This is like \code{,exit \cvar{exp}} except that the evaluation of \cvar{exp}
    is tail-recursive with respect to the command processor.  This
    means that the command processor itself can probably be GC'ed,
    should a garbage collection occur in the execution of \cvar{exp}.
    If an error occurs Scheme~48 will exit with a non-zero value.

\item \code{,translate \cvar{from} \cvar{to}}\\
    For \code{load} and the \code{,load} command
     (but not for \code{open-\{in|out\}put-file}), file
    names beginning with the string \cvar{from} will be changed so that the
    initial \cvar{from} is replaced by the string \cvar{to}.  E.g.
\begin{example}
\code{,translate /usr/gjc/ /zu/gjc/}
\end{example}
    will cause \code{(load "/usr/gjc/foo.scm")} to have the same effect as
    \code{(load "/zu/gjc/foo.scm")}.

\item \code{,from-file \cvar{filename} \cvar{form} \ldots ,end}\\
    This is used by the \code{cmuscheme48} Emacs library to indicate the file
    from which the \cvar{form}s came.  \cvar{Filename} is then used by the
    command processor to determine the package in which the \cvar{form}s
    are to be evaluated.
\end{description}

